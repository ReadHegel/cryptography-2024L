
%% Save the class definition of \subparagraph
\let\llncssubparagraph\subparagraph
%% Provide a definition to \subparagraph to keep titlesec happy
\let\subparagraph\paragraph
%% Load titlesec
\usepackage[compact]{titlesec}
\usepackage{sectsty}
\sectionfont{\fontsize{11}{12}\selectfont}


%\usepackage{titling}


%\titleformat*{\section}{\large\headingfont}

%% Revert \subparagraph to the llncs definition
\let\subparagraph\llncssubparagraph


%\titleclass{\ex}{straight}[\section]
%\newcounter{ex}

\usepackage{enumitem}

\usepackage[
n,
operators,
advantage,
sets,
adversary,
landau,
probability,
notions,
logic,
ff,
mm,
primitives,
events,
complexity,
asymptotics,
keys]{cryptocode}

\newenvironment{sol}{\par\noindent\rule{\textwidth}{0.4pt}
\paragraph{Solution:}}{\par\noindent\rule{\textwidth}{0.4pt}
}

\usepackage{hyperref}
\hypersetup{
    colorlinks,
    linkcolor={red!80!black},
    citecolor={blue!80!black},
    urlcolor={blue!80!black}
}

\usepackage{wrapfig}
\usepackage{framed}
\usepackage[margin=2.5cm]{geometry}


\newcommand{\half}{{\textstyle\frac{1}{2}}}

\newcommand{\SD}[2]{\Delta(#1 \: ;  #2)}

\newcommand{\cA}{{\mathcal A}}
\newcommand{\cO}{{\mathcal O}}
\newcommand{\cC}{{\mathcal C}}
\newcommand{\cB}{{\mathcal B}}
\newcommand{\cP}{{\mathcal P}}
\newcommand{\cR}{{\mathcal R}}
\newcommand{\cQ}{{\mathcal Q}}
\newcommand{\cF}{{\mathcal F}}
\newcommand{\cG}{{\mathcal G}}
\newcommand{\cE}{{\mathcal E}}
\newcommand{\cH}{{\mathcal H}}
\newcommand{\cJ}{{\mathcal J}}
\newcommand{\cM}{{\mathcal M}}
\newcommand{\cV}{{\mathcal V}}
\newcommand{\cS}{{\mathcal S}}
\newcommand{\cX}{{\mathcal X}}
\newcommand{\cY}{{\mathcal Y}}
\newcommand{\cZ}{{\mathcal Z}}
\newcommand{\cK}{{\mathcal K}}
\newcommand{\cI}{{\mathcal I}}
\newcommand{\cT}{{\mathcal T}}
\newcommand{\cW}{{\mathcal W}}


\usepackage{fullpage}


%\titleformat{\ex}{\normalsize\bfseries}{}{0em}{Exercise \theex:~}
%\titlespacing*{\ex}{0pt}{3.25ex plus 1ex minus .2ex}{1.5ex plus .2ex}

\usepackage{graphicx}
\usepackage{xcolor}

\newcommand{\DES}{{\mathsf{DES}}}
\newcommand{\negate}[1]{\overline{#1}}
\newcommand{\polyn}{\mathsf{poly}}
\newcommand{\Tag}{\mathsf{Tag}}

\newcommand{\PRNG}{\mathsf{PRNG}}

\newcommand{\suffix}{\mathsf{suffix}}
\newcommand{\CRT}{\mathsf{CRT}}
\newcommand{\puzzle}{\mathsf{puzzle}}

%\newcommand{\floor}[1]{\lfloor #1 \rfloor}
\newcommand{\bins}{\bin^*}
\newcommand{\conc}{\: || \:}

\usepackage{mdframed}

\mdfdefinestyle{protstyle}{%
        userdefinedwidth=.5\linewidth,
        align=center,
	linecolor=black,linewidth=1.5pt,%
	frametitlerule=true,%
	frametitlebackgroundcolor=gray!20,
	innertopmargin=\topskip,
        skipabove=12pt,
}

\newenvironment{prot}[1]{\begin{mdframed}[style=protstyle,frametitle={#1}]}{\end{mdframed}\vspace{.3cm}}

\newcommand{\DATE}[1]{\begin{flushright}\emph{[#1]}\end{flushright}}